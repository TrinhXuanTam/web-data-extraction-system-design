\subsection{Summary}
This system is designed to process and manage an infinite stream of web data efficiently, even on a machine with limited memory. Each module within the system plays a critical role in ensuring that data is handled efficiently from ingestion to presentation.

\subsubsection{Efficient Data Handling}
\begin{itemize}
    \item The \textbf{Data Ingestion Module} uses buffering and bulk insertion strategies to manage high volumes of incoming data without overloading the system's memory. This allows the system to handle bursts of data efficiently.
    \item The \textbf{Data Storage Module}, including both the file-based queue and the B+ tree, optimizes data storage and access patterns. The file-based queue minimizes overhead by storing data linearly and using a simple JSON serialization for persistence, while the B+ tree provides efficient indexing and retrieval capabilities for large datasets.
\end{itemize}

\subsubsection{Memory Optimization}
\begin{itemize}
    \item \textbf{Data structures and algorithms} used in the system are selected based on their low memory requirements and high efficiency. For instance, the B+ tree minimizes disk I/O by organizing data into blocks and storing indices in memory, which accelerates search operations without consuming excessive memory.
    \item The use of a file-based queue reduces the need for complex database management systems that typically require more memory and computational resources. This approach leverages the filesystem for data persistence, which is generally more memory-efficient and simpler to scale on a single machine.
\end{itemize}